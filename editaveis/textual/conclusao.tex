\chapter[Conclusão e Trabalhos Furutos]{Conclusão e Trabalhos Futuros}

Diante do estudo apresentado e dos resultados obtidos, pode-se observar que, de fato, houve uma redução no esforço necessário para integrar os incrementos de software enviado por cada um dos desenvolvedores para o repositório. Além disso, diante dos critérios de qualidade de teste estabelecidos, pode-se considerar que houve um aumento significativo na qualidade.

Sendo assim, os objetivos gerais do trabalho foram atingidos, visto que a ferramenta de integração contínua, Travis CI, foi utilizada para a verificar a integridade dos incrementos de software e para analisar os testes unitários, e a qualidade dos testes aumentou, pois a porcentagens de casos de testes passou de 64,3\% para 85,7\%.
