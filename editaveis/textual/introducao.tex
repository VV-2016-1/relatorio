\chapter[Introdução]{Introdução}

O TEM-DF (Tranparência de Escalas Médicas do DF) é um software que visa apresentar as escalas dos médicos da rede pública de saúde do Distrito Federal, facilitando encontrar algum médico da especialidade adequada e na região desejada, além de permitir aos usuários realizarem reclamações ou elogios aos médicos. 


O projeto foi desenvolvido por alunos da Faculdade do Gama - UnB, nas disciplinas de GPP e MDS e uma das restrições de qualidade do projeto é cobertura de testes acima de 90\%. Porém, por ter sido desenvolvido por programadores iniciantes, boa parte dos testes, mesmo os que passaram, não eram de boa qualidade, ou seja os casos não estão contemplando o particionamento de equivalência e os valores limites. Além de conter apenas testes no nível unitário e funcional. Outro problema, é que durante o desenvolvimento, muito código era enviado ao repositório oficial sem testes e esse código não foi barrado ao ser enviado para as branchs principais do projeto.


Esse software pode ser acessado por:

\url{https://github.com/EscalaSaudeDF/TEM-DF} 

E sua documentação pode ser acessada em:

\url{http://lappis.unb.br/redmine/projects/grupo-1-tem-transparencia-das-escalas-dos-medicos-do-df/wiki.}


\section{Problema}
\begin{itemize}
\item Falta de Verificação dos Testes unitários no momento da integração da build.
\item Baixa confiabilidade na qualidade dos testes unitários realizados.
\end{itemize}

\section{Objetivos Gerais }
\begin{itemize}
    \item Verificar a integridade, analisando se todos os testes de unidade e aceitação passaram, de uma build de software a ser adicionada
    \item Melhorar testes unitários realizados no software.
\end{itemize}

\section{Objetivos Específicos }
\begin{itemize}
    \item Aplicar uma ferramenta de Integração contínua
    \item Analisar os testes unitários no momento da integração
\end{itemize}
\section{Questões de Pesquisa}
\begin{itemize}
    \item Como pode ser validado um novo incremento de software?
    \item Como pode ser validado um novo incremento de software?
\end{itemize}
\section{Estratégia de Pesquisa}
\subsection{Revisão Sistemática de Literatura}
\begin{itemize}
    \item Verificação e Validação de software
    \item Testes de Integração contínua de software
    \item  Utilização da Ferramenta Travis em testes de integração contínua de software
    \item Padrões de Qualidade em Testes Unitários de software
\end{itemize}

\subsection{Metodologia}
\begin{itemize}
    \item Definição de Palavras-chave
    \item Geralçai de Uma String de Busca
    \item Aplicação de String na Base Scopos
    \item Filtro de Busca(Anális do título, Análise do Abstract, Referência)
    \item Snowball para trás
\end{itemize}

