\chapter[Introdução]{Introdução}

O TEM-DF (Tranparência de Escalas Médicas do DF) é um software que visa apresentar as escalas dos médicos da rede pública de saúde do Distrito Federal, facilitando encontrar algum médico da especialidade adequada e na região desejada, além de permitir aos usuários realizarem reclamações ou elogios aos médicos.


O projeto foi desenvolvido por alunos da Faculdade do Gama - UnB, nas disciplinas de GPP e MDS e uma das restrições de qualidade do projeto é cobertura de testes acima de 90\%. Porém, por ter sido desenvolvido por programadores iniciantes, boa parte dos testes, mesmo os que passaram, não eram de boa qualidade, ou seja os casos não estão contemplando o particionamento de equivalência e os valores limites. Além de conter apenas testes no nível unitário e funcional. Outro problema, é que durante o desenvolvimento, muito código era enviado ao repositório oficial sem testes e esse código não foi barrado ao ser enviado para as branchs principais do projeto.


Esse software pode ser acessado no seguinte \href{https://github.com/EscalaSaudeDF/TEM-DF}{endereço}%
\footnote{https://github.com/EscalaSaudeDF/TEM-DF}

E sua documentação pode ser acessada neste \href{http://lappis.unb.br/redmine/projects/grupo-1-tem-transparencia-das-escalas-dos-medicos-do-df/wiki}{endereço}%.
\footnote{http://lappis.unb.br/redmine/projects/grupo-1-tem-transparencia-das-escalas-dos-medicos-do-df/wiki}


\section{Problema}
\begin{itemize}
\item Falta de Verificação dos Testes unitários no momento da integração da build.
\item Baixa confiabilidade na qualidade dos testes unitários realizados.
\item Excesso de esforço para integrar as builds de software
\item Falhas no software, mesmo diante a cobertura acima de 90\%
\end{itemize}

\section{Objetivos Gerais}
\begin{itemize}
    \item Verificar a integridade, analisando se todos os testes de unidade passaram, de uma build de software a ser adicionada
    \item Melhorar a confiabilidade dos testes unitários realizados no software.
\end{itemize}

\section{Objetivos Específicos }
\begin{itemize}
    \item Aplicar uma ferramenta de Integração contínua
    \item Analisar os testes unitários no momento da integração
    \item Melhorar a qualidade dos testes unitários implementados
\end{itemize}
\section{Questões de Pesquisa}
\begin{itemize}
    \item Como pode ser validado um novo incremento de software no nível testes unitários?
    \item Como garantir melhor qualidade dos testes unitários realizados?
\end{itemize}
\section{Estratégia de Pesquisa}
\subsection{Revisão Sistemática de Literatura}
\begin{itemize}
    \item Verificação e validação de software
    \item Integração contínua de software
    \item  Utilização da ferramenta travis para integração contínua de software
    \item Padrões de qualidade em testes unitários de software
\end{itemize}

\subsection{Metodologia}
\begin{itemize}
    \item Definição de palavras-chave
    \item Gerar de uma string de busca
    \item Aplicação de string na base scopus
    \item Filtro de susca(análise do título, análise do abstract, referência)
    \item Snowball para trás
\end{itemize}

\section{Resultados Esperados}
\subsection{Abordagem}
\begin{itemize}
    \item Qualitativas, tendo a confiabilidade de que os incrementos de software estão sendo testados corretamente.
    \item Quantitativa, o projeto deverá ter pelo menos 90\% de cobertura de código de testes.
\end{itemize}

\subsection{Métodos}
\begin{itemize}
    \item Aplicação de uma política de avaliação de qualidade estática de código contínua para validação de cada submissão de incremento de software.
\end{itemize}

\subsection{Técnicas}
\begin{itemize}
    \item Integração Contínua
\end{itemize}

\subsection{Ferramentas}
Ferramentas a serem utilizadas e formas de estudos empíricos (Estudos de caso, questionários, entrevistas, estudos estatísticos, dentre vários outros possíveis).
\begin{itemize}
  \item \href{https://docs.travis-ci.com}{Travis}\footnote{https://docs.travis-ci.com}: Ferramenta utilizada para os estudos empíricos.
    \item Estudo de Caso: Tecnologia de metodologia de pesquisa.
\end{itemize}
