\chapter[Metodologia]{Metodologia}

Este capítulo tem como objetivo descrever como será dirigida a pesquisa.
Ela será dividida em três fases, que serão explicadas em linhas gerais a seguir,
e para auxiliá-la, um protocolo dos princípios de revisão sistemática será detalhado.

A fim de responder as questões de pesquisa, o estudo deve atingir todos
os objetivos do trabalho. Porém, nessa fase do estudo, será utilizada uma revisão
sistemática para responder as questões relacionadas aos dois objetivos
escolhidos. São eles:

\begin{itemize}
\item Verificar a integridade, analisando se todos os testes de unidade e aceitação
passaram, de uma build de software a ser adicionada terceirização;

\item Melhorar a confiabilidade dos testes unitários realizados no software.
\end{itemize}

A pesquisa para responder os objetivos acima terá as três fases:
identificação do tema e criação do plano de pesquisa, embasamento teórico e
execução do estudo de caso. De maneira com que cada fase seja
responsável por um conjunto de atividades semelhantes.

Identificação do tema e criação do plano de pesquisa: para possibilitar a
resposta dos objetivos de pesquisa, será executada um estudo de caso
selecionando estudos primários que contribuam para a resposta da questão de
pesquisa. Para tanto, na execução de uma boa revisão sistemática se faz
necessário o uso de um protocolo predefinido, que é o conjunto de passos que
deverão ser seguidos, afim de agregar maior qualidade ao resultado final da
pesquisa.

Embasamento teórico: com a finalidade de conhecer sobre o tema que se
deseja estudar, é elaborado um referencial teórico a partir de estudos já
realizados em determinada área. Este, tem como objetivo solidificar uma base de
conhecimentos já estudados e comprovados das áreas que circundam o tema do
presente trabalho.

Execução dos princípios de revisão sistemática: de acordo com o plano de revisão
estabelecido previamente, será executada uma revisão sistemática simples,
com o intuito
de atingir as metas propostas para essa fase. A revisão será uma busca por
estudos importantes que auxiliem a resposta das questões do trabalho, executada
de maneira sistemática. Serão encontrados trabalhos de acordo com um padrão
(String de busca) e, a partir deles, será feita uma seleção dos dados.

As próximas seções apresentam o detalhamento do protocolo de revisão
sistemática.
