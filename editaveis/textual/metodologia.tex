\chapter[Metodologia]{Metodologia}

Este capítulo tem como objetivo descrever como será dirigida a pesquisa.
Ela será dividida em três fases, que serão explicadas em linhas gerais a seguir,
e para auxiliá-la, um protocolo dos princípios de revisão sistemática será
detalhado.

A fim de responder as questões de pesquisa, o estudo deve atingir todos
os objetivos do trabalho. Porém, nessa fase do estudo, será utilizado princípios
de revisão sistemática como embasamento teórioco para o estudo de caso, a fim de
responder as questões relacionadas aos dois objetivos
escolhidos. São eles:

\begin{itemize}
    \item Verificar a integridade, analisando se todos os testes de unidade e
        aceitação
        passaram, de uma build de software a ser adicionada;
    \item Melhorar a confiabilidade dos testes unitários realizados no software.
\end{itemize}

A pesquisa para responder os objetivos acima terá as três fases:
identificação do tema e criação do plano de pesquisa, embasamento teórico e
execução do estudo de caso. De maneira com que cada fase seja
responsável por um conjunto de atividades semelhantes.

Identificação do tema e criação do plano de pesquisa: para possibilitar a
resposta dos objetivos de pesquisa, será executada um estudo de caso
aplicando algumas técnicas que serão posteriormente discutidas.
Para tanto, será necessário executar alguns princípios de revisão sistemática
com intuito de gerar embasamento teórico para a execução do estudo de caso.
Dessa forma, se faz necessário o uso de um protocolo predefinido, que é o
conjunto de passos que deverão ser seguidos, afim de agregar maior
qualidade ao resultado final da pesquisa.

Execução dos princípios de revisão sistemática: de acordo com o plano de revisão
estabelecido previamente, será executada uma revisão sistemática simples,
com o intuito
de atingir as metas propostas para essa fase. A revisão será uma busca por
estudos importantes que auxiliem a resposta das questões do trabalho, executada
de maneira sistemática. Serão encontrados trabalhos de acordo com um padrão
(String de busca) e, a partir deles, será feita uma seleção dos dados.

Execução do estudo de caso: Primeiramente será aplicado o conceito de integração
contínua do software com base no referêncial adotado. Após, iremos melhorar a
confiabilidade dos testes executando os passos estabelecidos por \cite{e07}.
Primeiramente iremos fazer uma medição da confiabilidade dos testes tal como o
software está. Após, iremos aplicar as técnicas que serão definidas e executar
novamente a medição, avaliando se houve o não uma melhoria e de quanto.
As próximas seções apresentam o detalhamento do protocolo de revisão
sistemática.

\section{Protocolo de Princípios de Revisão Sistemática}
O protocolo será um planejamento detalhado do processo de revisão
sistemática que será tratado a seguir.
\subsection{Seleção das Fontes}
\label{sub:Seleção das Fontes}
Será utilizada para a revisão sistemática algumas bases de busca
automática. A seguinte base de dados será adotada para a pesquisa, pois
esta indexa 90\% dos artigos das outras bases (IEEE, SD, ACM, Compendex):

\begin{itemize}
    \item Scopus.
\end{itemize}

\subsubsection{Critério de Aceitação das Fontes}
\label{sub:Critério de Aceitação das Fontes}
Como critério de seleção das fontes, será analisada algumas das bases
que \cite{dyba} utilizou para escrever um estudo sobre desenvolvimento de
software. \cite{dyba} utiliza várias bases, entretanto, algumas não foram
utilizadas,
como a IEEE, por exemplo, pois os seus artigos são indexados na Scopus.

\subsection{Idioma dos Estudos}
\label{sub:Idioma dos Estudos}
Os estudos analisados deverão estar escritos em inglês ou português. O idioma da
língua inglesa foi escolhido pois abrange grande quantidade de estudos da área
de tecnologia da informação. O português foi escolhido por ser a nossa língua
mãe.

\subsection{Método de Busca das Publicações}
\label{sub:Método de Busca das Publicações}
Para a obtenção dos estudos, serão utilizadas a busca manual e a busca
automática em bibliotecas digitais.

\subsection{Expressões Gerais de Busca}
\label{sub:Expressões Gerais de Busca}
Para a busca eletrônica, foram utilizadas três grupos de strings de busca.
Uma para cada tema que necessitávamos para elaborar o embasamento teórico.
Dessa maneira, foram formadas as seguintes strings:

\subsubsection{Verificação e Validação}
\label{sub:Verificação e Validação}
Com intuito de dar embasamento teórico para a área de Verificação e validação,
foi estabelecida uma pesquisa na Scopus seguindo os seguintes itens e as
seguinte string de busca.
\begin{enumerate}
    \item Product
    \item Software
    \item Verification
    \item Validation
    \item Quality
\end{enumerate}
Dessa maneira, a String de busca formada para a pesquisa foi a seguinte:
\\
\\
\centerline{\textbf{1 AND 2 AND 3 AND 4 AND 5}}

\subsubsection{Integração Contínua}
\label{sub:Integração Contínua}
A primeira string abaixo selecionou 1440 artigos, sendo 3 escolhidos para a
avaliação, sendo que a segunda retornou 110 e não teve nenhum aproveitado. A
terceira é exatamente igual a segunda, porém, foi aplicada no portal da Capes e
retornou 24 estudos, sendo que destes, um foi utilizado:

\\
\\
\centerline{\textbf{1 - "continuous integration" and software}}
\\
\centerline{\textbf{2 - "continuous integration" and tools}}
\\
\centerline{\textbf{3 - "continuous integration" and tools}}


\subsubsection{Qualidade dos Testes}
\label{sub:Qualidade dos Testes}
Para intuir sobre qualidade de testes, precisávamos de estudos que tratassem
sobre confiabilidade destes no ambiente de desenvolvimento de software. Dessa
maneira, algumas palavras se fizeram essenciais na obtenção dos resultados
esperados. Elas foram:

\begin{enumerate}
    \item Software Engineering
    \item Software
    \item Softwares
    \item Testing
    \item Test
    \item Confirmation
    \item Quality
    \item Reability
\end{enumerate}

Dessa maneira, a String de busca formada para a pesquisa foi a seguinte:
\\
\\
\centerline{\textbf{(1 OR 2 OR 3) AND (4 OR 5 OR 6) AND (7 OR 8)}}


\subsection{Procedimentos de Seleção e Critérios de Avaliação}
\label{sub:Procedimentos de Seleção e Critérios de Avaliação}

Para a inclusão dos estudos encontrados nas buscas, os seguintes
critérios constituirão a lista:

\begin{itemize}
    \item Os artigos são da área da computação;
    \item Os artigos abrangem o contexto de qualidade de testes ou integração
        contínua em uma equipe de desenvolvimento de software;
    \item Os artigos que são estudos primários ou secundários;
    \item Os artigos estão disponíveis gratuitamente pela rede da UnB;
\end{itemize}

Serão retirados da lista todos os artigos que não se enquadram nos
critérios a seguir:

\begin{itemize}
    \item Os artigos são de equipes de desenvolvimento de software;
    \item Os artigos trazem estudos sobre equipes de desenvolvimento que
        utilizam métodos iterativos;
    \item Os artigos trazem a aplicação ou avaliação de um ou mais
        métodos de avaliação de qualidade de testes ou integração contínua
        em uma equipe de desenvolvimento de software;
    \item Os estudos estão completamente escritos em inglês ou português.
\end{itemize}

\subsection{Procedimentos de Seleção}
\label{sub:Procedimentos de Seleção}

Após a inclusão dos artigos, os seguintes critérios de qualidade e
importância irão ser aplicados para a seleção dos estudos, afim de certificar que
irão atender a questão de pesquisa:

\begin{itemize}
    \item Os artigos serão avaliados por meio da leitura do título e do abstract;
    \item Os artigos serão avaliados por meio da leitura do estudo na sua
        totalidade e, passando por uma análise crítica;
    \item Os artigos devem trazer consigo um conhecimento que auxilie na
        resposta da questão de pesquisa;
\end{itemize}
