\chapter[Referencial]{Referencial Teórico}

\section{Verificação e validação}
\section{Qualidade de Testes}
Quando se trata de manufatura ou produção de componentes físicos, os testes so
bre a confiabilidade do que foi produzido é feito a partir de uma porcentagem
do lote retirado do todo. Caso a quantidade returada do lote passe nos testes,
ele é aprovado, caso não, todas peças são rejeitadas.

Ao se tratar de software, a maneira de executar os testes é bastante diferente.
Cada software tem a sua peculiariedade e é desenvolvido de acordo com as
capacidades dos desenvolvedores participantes do projeto. (Aditya), em seu estudo,
afirma que antigamente eram realizados os testes com base nos testes de hardware.
E, que esses, por sua vez, não eram realísticos.

Os softwares são desenvolvidos e uma bateria de testes primeiramente é definida no
documento de casos de teste. Vários testes são  executados de acordo com o sistema,
podendo ser entrem eles testes de:

\begin{itemize}
\item Testes Unitários
\item Testes de Integração
\item Testes de Sistema
\item Testes de Aceitação
\item Testes de Instalação
\end{itemize}

Mas uma pergunta que sempre entrigou os programadores de teste é:
\section{Integração Contínua}

